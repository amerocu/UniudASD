% !TEX root = ../teoria.tex

\chapter{B-Tree}

\section{Definizione}

Un B-tree di grato \( t > 2 \) e un albero t.c.
\begin{enumerate}
	\item Ogni nodo \( x \) contiene \( n[x] \) chiavi ordinate.
	\item Ogni nodo \( x \), non foglia, ha \( n[x]+1 \) figli.
	\item Ogni chiave \( k \), contenuta nel sottoalbero \( C_i \) appartenente al nodo \( z \) e compresa tra le chiavi \( key_{i-1}[z] \) e \( key_{i}[z] \).
	\item Tutte le foglie dell'albero si trovano allo stesso livello.
	\item Per ogni nodo x
	\begin{enumerate}
		\item \( x \neq root[T] \Rightarrow t-1 \leq n[x] \leq 2t - 1 \)
		\item \( x = root[T] \Rightarrow 1 \leq n[x] \leq 2t - 1 \)
	\end{enumerate}
\end{enumerate}

\section{Operazioni}
\subsection{Inserimento}
\subsection{Cancellazione}