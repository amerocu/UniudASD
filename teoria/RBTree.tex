% !TEX root = ../teoria.tex

\chapter{RB-Tree}

\section{Definizione}

\begin{enumerate}

	\item Ogni nodo ha un campo color, che può assumere valore "red" oppure "black".
	\item Tutte le foglie sono "NIL" e "black" (non hanno chiave).
	\item Ogni nodo "red" ha due figli "black".
	\item Per ogni nodo x, tutti i cammini x->foglia attraversano sempre lo stesso numero di nodi black.
	\item[Bis.] La radice di un RB-Tree è "balck"

\end{enumerate}

\section{Operazioni}
\subsection{Rotazioni}
\subsection{Inserimento}

\begin{enumerate}
	\item Inserisco il nuovo elemento come foglia rossa.
	
	\item padre "nero"
	\item padre "rosso"
	\begin{enumerate}
		\item zio "rosso"
		\item zio "nero"
		\begin{enumerate}
			\item stesso lato
			\item lato diverso
		\end{enumerate}
	\end{enumerate}
\end{enumerate}

\subsection{Cancellazione}

\begin{enumerate}
	\item Sostituisco l'elemento da cancellare con il suo predecessore/successore.
	
	\item Elimino il predecessore/successore, sostituendolo con il suo UNICO figlio.
	
	\item predecessore/successore sono figlio:
	
	\begin{enumerate}
		\item Almeno uno rosso, OK
		\item Entrambi neri, il figlio diventa "double-balck"
		\item il fratello e:
		\begin{enumerate}
			\item rosso: rotate sul padre, così che il fratello diventi nero. 
			\item nero con figli neri: sposto l'altezza nera dal nodo e dal fratello, al padre, se il padre era nero ritorsione, sennò OK.
			\item nero con figlio opposto rosso: il figlio concorde diventa il fratello
			\item nero con figlio concorde rosso: il figlio rosso diventa il fratello (sposto il rosso da concorde ad opposto) OK
		\end{enumerate}
	\end{enumerate}
	
\end{enumerate}















