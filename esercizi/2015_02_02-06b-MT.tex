% !TEX root = ../main.tex

\subsection[06b MT]{Esercizio 06b Marco Turchetto}
\begin{enumerate}[label=\alph*)]
\item L'equazione ricorsiva: 
	\begin{equation*}
		MC(n)=\begin{cases}
			\BigTheta{1}				& \text{se $n +1  < 4$}.\\
			2MC(n - 3) + \BigTheta{1}	& \text{altrimenti}.
 		\end{cases}
		= \BigTheta{2^n}
	\end{equation*}
	
	
	
\item Al termine il vettore non è ordinato e si può facilmente provare ordinando il vettore $ \{ 5, 4, 3, 2, 1 \} $
	
	\begin{center}
		\begin{tabular}{*{5}{|c}|l}
			\noborder{$1$} & \noborder{$2$} & \noborder{$3$} & \noborder{$4$} & \noborder{$5$} & Indici \\ \cline{1-5}
			$5$ & $4$ & $3$ & $2$ & $1$ & Vettore iniziale \\ \cline{1-5}
			$4$ & $5$ &     &     &     & MC(1,2) \\ \cline{1-5}
			    &     &     & $1$ & $2$ & MC(4,5) \\ \cline{1-5}
			$4$ & $5$ & $3$ & $1$ & $2$ & MC(1,5) \\ \cline{1-5}
		\end{tabular}
	\end{center}
\end{enumerate}
