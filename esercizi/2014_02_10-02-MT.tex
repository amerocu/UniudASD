% !TEX root = ../main.tex

\subsection[02 MT]{Esercizio 02 Marco Turchetto}

\begin{enumerate}[label=\alph*)]
\item 
	\lstinputlisting{./esercizi/2014_02_10-02-MT-code.txt}
	
	\begin{equation*}
		T(n)=
		\begin{cases}
			\BigO{1}										& \text{se n = 1} \\
			2T \left(\frac{n}{2} \right) + \BigO{\log n}	& \text{se n > 1} \\
		\end{cases}
		\Rightarrow \BigO{\log ^2 n}
	\end{equation*}
	
\item
	Un algoritmo con tale complessità non può esistere perché con tale algoritmo
	sarebbe possibile ordinare un vettore con complessità \( \BigO{\log ^2 n} \)
	palesemente in contrasto con il limite inferiore di \( \BigOmega{n \log n} \).
\end{enumerate}