% !TEX root = ../main.tex

\subsection[01a MT]{Esercizio 01a Marco Turchetto}

\begin{enumerate}[label=\alph*)]
\item
	Definizioni di \( \BigO{g(n)} \) e \( \BigOmega{h(n)} \) \\
	\begin{align*}
		\BigO{g(n)} &= \{ f(n) | \exists \tilde{n}, c_1 s.t. f(n) \leq c_1 g(n) \forall n > \tilde{n} \} \\
		\BigOmega{h(n)} &= \{ f(n) | \exists \tilde{n}, c_1 s.t. f(n) \geq c_1 h(n) \forall n > \tilde{n} \} \\
	\end{align*}

\item 
	Dimostrare per induzione che \( T(n) \in \BigO{\sqrt{n} \log n} \):

	\begin{equation*}
		T(n)=
		\begin{cases}
			\BigO{1} \\
			2T \left(\frac{n}{4} \right) + c \sqrt{n} \\
		\end{cases}
	\end{equation*}
	
	Caso base con $\tilde{n} = 4$:
	
	\begin{align*}
		T(4) &= 2T \left(\frac{4}{4} \right) + c \sqrt{4} \\
			 &= 2 \BigO{1} + 2c = 2c_1 + 2c_2 \\
	\end{align*}
	
	\( \exists c_3 | T(n) \leq c_3 \sqrt{n} \log n \ t.s. \ n = \tilde{n} \) ? \\
	
	\begin{align*}
		2c_1 + 2c_2 &\leq c_3 \sqrt{4} \log 4 \\
		\frac{c_1 + c_2}{\log 4} &\leq c_3 \Rightarrow \exists c_3 > 0
	\end{align*}
	
	Ipotesi induttiva:\\
	\[
		IP \ T(r) \in \BigO{\sqrt{n} \log n} \forall r | \tilde{n} < r < n
	\]
	
	Tesi:\\
	\[
		TS \ T(n) \in \BigO{\sqrt{n} \log n} \forall n | n \geq \tilde{n}
	\]
	
	\begin{align*}
		T(n) &= 2T \left( \frac{n}{4} \right) + c\sqrt{n}  \\
		T(n) &\leq 2c_2\sqrt{\frac{n}{4}} \log \frac{n}{4} + c_1\sqrt{n}  & \text{Per ipotesi induttiva} \\
	\end{align*}
	
	\( \exists d | T(n) \leq d \sqrt{n} \log n \ \forall n > \tilde{n} \) ? \\
	
	\begin{align*}
		T(n) \leq 2c_2\sqrt{\frac{n}{4}} \log \frac{n}{4} + c_1\sqrt{n} &\leq d \sqrt{n} \log n \\
		% Sarebbe da aggiungere qualche passaggio
		d \log 4 &\geq c \\
		\exists d > 0 | d &\geq \frac{c}{\log 4} \\
	\end{align*}

\end{enumerate}
