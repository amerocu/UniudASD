% !TEX root = ../main.tex

\subsection[03a MT]{Esercizio 03a Marco Turchetto}

\begin{enumerate}[label=\alph*)]
%\item
%	\lstinputlisting[firstline=1, lastline=9]{./esercizi/2014_02_10-03a-MT-code.txt}
	
\item
	La funzione \texttt{RetriveMin} restituisce l'indice dell'elemento più piccolo eseguendo una ricerca lineare su tutte le foglie della heap.
	Questo è il metodo più efficiente considerando le ipotesi, ed ha complessità \BigTheta{n}.
	
	\lstinputlisting[firstline=11, lastline=22]{./esercizi/2014_02_10-03a-MT-code.txt}
	
\item
	La procedura \texttt{DeleteMin} prima cerca il minimo tramite la funzione \texttt{RetriveMin} e poi elimina l'elemento tramite la procedura \texttt{Delete} propria delle max/min-Heap.
	La complessità è la somma delle complessità delle due funzioni chiamate \( \BigTheta{n} + \BigO{ \log n} = \BigTheta{n}\)
	
	\lstinputlisting[firstline=24, lastline=27]{./esercizi/2014_02_10-03a-MT-code.txt}
	
\item
	La complessità nel caso peggiore è in quello minore è \BigTheta{n} che equivale alla componente associata alla ricerca del minimo, nonché quella prevalente, che è indipendente da ipotesi sulla distribuzione delle chiavi.
	
\item
	La complessità nel caso minore e peggiore è \BigTheta{\log n}, questo perché nella procedura \texttt{InsertKey(H, m)}, l'elemento m viene inserito come ultima foglia, ed essendo \( m > max(H) \) l'elemento dovrà sempre risalire l'intera heap, per diventarne la radice.
	
	\lstinputlisting[firstline=1, lastline=10]{./esercizi/2014_02_10-03a-MT-code.txt}
		
\end{enumerate}






